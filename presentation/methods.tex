

%Slide Preparation
\frame
{
  \frametitle{Preparaci\'on de la Muestra}
  \begin{columns}
\begin{column}{0.6\textwidth}
   \begin{itemize}
      \item<1-> Hacemos un Sandwich de $Al$,$Al_2O_3$ y $Pb$
      \item<2-> $Pb$ tiene $T_C$ mas alto, asi tenemos una union Super-Normal
      \item<2-> $Al_2O_3$ es el aislante, causa la bareda tunnel.
      \item<3-> Preparacion por evaporacion en un vacio.
      \item<4-> Tenemos varias  uniones en cada cristal.
  \end{itemize}
  \end{column}
\begin{column}{0.4\textwidth}
	\begin{figure}[!h] \label{sample}
	\includegraphics<4->[width=\textwidth]{sample}
	\end{figure}
\end{column}


\end{columns}
  
  %grafica de la muestra(show at 4?)
}



%thickness estimation

\frame
{
  \frametitle{Estimando las Anchuras de Capas}
  
  \begin{itemize}
     \item<1-> 2 Metodos:
        
        \begin{enumerate}
           \item<2-> Ley de Ohm: $h=\frac{\rho l}{Rb}$
           \item<3-> Suponiendo evaporaci\'on isotropa: $h=\frac{V}{4\pi r}$
        \end{enumerate}
        
     \item<4-> Metodo 1 no funciona, anchura no es bastante homogena.
     \item<5-> Anchura y resistencia del aislante se controla por tiempo al aire.
     \begin{figure}[!h] \label{sample}
	\includegraphics<6->[width=0.5\textwidth]{isolator}
     \end{figure}

     %Table from Giaever?      
  \end{itemize}

  
  %grafica? o solo formulas?
 
}

%Cryostat

\frame
{
  \frametitle{Enfriando la Muestra}
   \begin{columns}
   \begin{column}{0.6\textwidth}

   \begin{itemize}
     \item<1->Necesitamos $T$ entre $T_{C,Pb}=7.193 K$ y $T_{C,Al}=1.140 K$
     \item<2->Ponemos la muestra en un criostato con $He$ liquido,
     \item<3->A $p_{\text{ambiente}}$ $He$ liquido tiene $T=4.2 K$
     \item<4->Podemos bajar $T$ bajando la presion en el criostato
   \end{itemize}
   
     \end{column}
     \begin{column}{0.4\textwidth}
	\begin{figure}[!h] \label{sample}
	\includegraphics<1-3>[width=\textwidth]{cryostat}
	\includegraphics<4->[width=\textwidth]{vap_he}
     \end{figure}
\end{column}
\end{columns}
   
   
  %grafica criostato
  %Vapor pressure
}

%Measurement
  %4 teminales
  
  
\frame
{
  \frametitle{Tomando los Datos}
     \begin{columns}
   \begin{column}{0.6\textwidth}

  \begin{itemize}
    \item<1-> Solo nos interesa la conductancia de la union, no la de los cables.
    \item<2-> Medimos con 4 terminales
    \item<3-> Medimos a intervalos fijos de intensidad.
    \item<4-> Rangos y resoluciones differentes.
  \end{itemize}
  
       \end{column}
     \begin{column}{0.4\textwidth}
	\begin{figure}[!h] \label{sample}
	\includegraphics<2->[width=\textwidth]{4term}
     \end{figure}
\end{column}
\end{columns}

%4 term pic from wikipedia?
}