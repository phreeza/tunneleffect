The measurement at the lowest temperature that we could perform gives qualitatively different results. In fig. (figure???) can be seen the appearance of new effects. The transition temperature $T_c$ for Aluminum is $7.193 K$, and we have established experimentally the temperature for data as $1.2 K$. However, due to the difference between bulk properties and those that we must consider here for films, the $T_c$ is a little bit greater (REFERENCE??).

The situation is qualitatively clear now, i.e. the temperature has fallen enough to let some portions of Aluminum change to superconducting state, creating this combined effect between normal-superconductor and superconductor-superconductor junctions. It can be explained by considering a sum of junctions of these two kinds in parallel, so as to produce an I-V curve with peaks inside the gap of the Aluminum. For a superconductor-superconductor junction appear two peaks, located at $| \Delta_1-\Delta_2|$ and $(\Delta_1+\Delta_2)$. In fig. �? this can be seen...

