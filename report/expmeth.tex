In order to measure the effect described above in the case of normal-superconductor tunneling, we prepare samples of two metals with different transition temperatures separated by an insulator. We vapor-deposited thin layers of aluminum and lead on a microscope glass slide,  leaving the Aluminum layer at the open air for a short period to let some insulating $AlO_2$ oxide form.

IT HAS TO BE CLEAR THAT WE HAVE A SANDWITCH!!!!!!!

1) sample preparation: vacuum chamber (torr?? why?? mean free path), filament, layer thickness (method for calculating it: isotropy or resistance? Justify that the second one gives smaller thickness with Poisson distribution, because we have rare events), ... HOW MUCH/MANY?????!!!!!!!!!!!!!!

2) Cryostat, nitrogen, helium, vacuum pump, manometer, T-P of vapor-pressure He, why don't we have different P's up and down in the cryostat? The T is different... And... HOW MUCH???

3) Measurement: 4 terminals (why? HOW MUCH?), constant steps sized intensity, 