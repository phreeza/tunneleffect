
\documentclass[twocolumn, twoside,a4paper,10pt]{article}
\pagestyle{plain}
\addtolength{\textheight}{2cm}
\includeonly{expmeth}

\usepackage{amsmath, amsthm, amsfonts}
\usepackage{graphicx}
\usepackage[show]{ed}

%---------------------------------------------------------------------
\title{\textbf{Superconductivity and Electron Tunneling}}
\author{Thomas McColgan and Miguel Garc\'ia Echevarr\'ia\\
	\small{\textit{Laboratorio de Bajas Temperaturas, Dpto. de F�sica de la Materia Condensada}} \\
	\small{\textit{Facultad de Ciencias, UAM}}
	}
\date{Madrid, \today}
%-----------------------------------------------------------------
\begin{document}
%-----------------------------------------------------------------
%-----------------------------------------------------------------
%ABSTRACT
\twocolumn[ 
\begin{@twocolumnfalse} 
\maketitle % need full-width title 
\begin{abstract} 
\small{We repeat the 1960 experiment by Giaver (\cite{giaever1}, \cite{giaever2}, \cite{giaever3} and \cite{frerichs}), in which the tunneling current from a superconductor through an insulating film is measured to obtain the width of its gap. Our results and the discrepancies to the BCS Theory \cite{bcs} are discussed. We measure the lead energy gap to be $(1.4 \pm 0.1)$ meV for $T<4.2\ K$}.
\vspace{1cm}
\end{abstract}
\end{@twocolumnfalse}
]

%-----------------------------------------------------------------
%-----------------------------------------------------------------
\section{Introduction}

% === Introduction ===

This is the introduction! Yes!

%Motivation(BCS predicts Gap)
The theory of superconductivity presented by Bardeen, Cooper and Schrieffer in 1957 predicts a gap in the allowed energies for electrons. Even though this gap had been measured indirectly by several methods, it had not been measured directly until Giaver conducted his electron tunneling experiment, which we have attempted to reproduce here.

%Historical Context(Giaver, Nobel)



%-----------------------------------------------------------------
%-----------------------------------------------------------------
\section{Theoretical Approach}
%---------------------------------------------------------------------------------------
\begin{frame}
\frametitle{Efecto t\'unel cu\'antico}

\begin{figure}[h!]
\centering
\includegraphics[width=0.5\textwidth]{tunnelling}
\end{figure}

Coeficiente de transmisi\'on: $T \propto e^{-d}$ \hspace{0.5cm} si $\rho d >>1$, $\rho=\sqrt{\frac{2m(V_0-E)}{\hbar^2}}$
\vspace{0.2cm}

\pause
\begin{center}
Pero tenemos algo m\'as complejo... \textcolor{red}{una transici\'on discreto $\rightarrow$ cont\'inuo!!}

$\Downarrow$

\textcolor{blue}{La Regla de Oro de Fermi}
\end{center}

\end{frame}
%---------------------------------------------------------------------------------------
%---------------------------------------------------------------------------------------
\begin{frame}
\frametitle{Gr\'afico ilustrativo}

\begin{figure}[!h] \label{fermi_levels2}
\includegraphics[width=0.5\textwidth]{fermi_levels2}
\end{figure}


\end{frame}
%---------------------------------------------------------------------------------------
%---------------------------------------------------------------------------------------
\begin{frame}
\frametitle{C\'alculo de I(V) con la Regla de Oro de Fermi 1}

1 \textcolor{blue}{La tasa de transici\'on $W$ de un $e^-$ para pasar de una estado ocupado en la l\'amina 1 a un estado vac\'io de la l\'amina 2:}
\begin{equation*}\label{probability1}
W_{1\to 2}^{1e} = \frac{2\pi}{\hbar} |T_{21}|^2 f(\epsilon_{\mathbf{p_1}}) [1-f(\epsilon_{\mathbf{p_2}})]
		\delta(\epsilon_{\mathbf{p_1}}-\epsilon_{\mathbf{p_2}})\delta_{s_1s_2},
\end{equation*}

\pause
2 \textcolor{blue}{Asumiremos que el hamiltoniano no depende del \emph{esp\'in} y que acopla d\'ebilmente los eletrodos si el potencial aplicado es peque\~no respecto a $\epsilon_F$.}
$$V < 3meV << 10eV \approx \epsilon_F$$
\textcolor{blue}{As\'i que la tasa total de transici\'on del electrodo 1 al 2 ser\'a:}

\begin{equation*}\label{probability3}
W_{1\to 2} = \frac{4\pi}{\hbar} |T|^2 \sum_{\mathbf{p_1},\mathbf{p_2}}  
		f(\epsilon_{\mathbf{p_1}}) [1-f(\epsilon_{\mathbf{p_2}})] 
		\delta(\epsilon_{\mathbf{p_1}}-\epsilon_{\mathbf{p_2}}).
\end{equation*}

\pause
%\vspace{0.2cm}
\begin{center}
\emph{\textcolor{red}{La tasa de transici\'on en la direcci\'on opuesta es an\'aloga}}
\end{center}

\end{frame}
%---------------------------------------------------------------------------------------
%---------------------------------------------------------------------------------------
\begin{frame}
\frametitle{C\'alculo de I(V) con la Regla de Oro de Fermi 2}

\textcolor{blue}{Expresi\'on para la corriente en la direcci\'on $1\to 2$:}
\begin{equation*}\label{current1}
I = e\ (W_{1\to 2} - W_{2\to 1}),
\end{equation*}

\pause
\begin{equation*}\label{current2}
I = \frac{4\pi e}{\hbar} |T|^2 \sum_{\mathbf{p_1},\mathbf{p_2}}  
		[f(\epsilon_{\mathbf{p_1}})-f(\epsilon_{\mathbf{p_2}})] 
		\delta(\epsilon_{\mathbf{p_1}}-\epsilon_{\mathbf{p_2}})
\end{equation*}

\pause
\textcolor{blue}{Momentos $\sim$ cuasicont\'inuo $\Rightarrow$ sumatorios $\sim$ integrales}

\textcolor{blue}{Voltaje V $\Rightarrow$ $\mu_2-\mu_1=eV$}

\begin{equation*}\label{current3}
I = \frac{4\pi e}{\hbar}  |T|^2 \int_{-\infty}^{\infty} d\epsilon\ N_1(\epsilon-eV)\ N_2(\epsilon) [f(\epsilon-eV)-f(\epsilon)]
\end{equation*}


\end{frame}
%---------------------------------------------------------------------------------------
%---------------------------------------------------------------------------------------
\begin{frame}
\frametitle{Densidad de estados para los 3 tipos de uniones}

\begin{columns}
\begin{column}{0.4\textwidth}
	\begin{figure}[!h] \label{fermi_levels}
	\includegraphics[width=\textwidth]{fermi_levels}
	\end{figure}
\end{column}
\begin{column}{0.6\textwidth}
	\begin{itemize}[<+->]
	\item{Estado normal: $N_N(\mu)$}
	\item{Para un $V$ peque\~no, $N_N(\mu)\simeq cte$}
	\item{Densidad de estados para superconductor:
		\begin{equation*}\label{ns}
		N_S(E) = 
		\left\{ 
		\begin{array}{ll} 
		N_N(\epsilon)\frac{|E|}{\sqrt{E^2-\Delta^2}},	&	|E| > \Delta 	\\ 
		0,								& 	|E| \leq \Delta	\\
		\end{array}
		\right.
		\end{equation*}
		}
	\end{itemize}
\end{column}
\end{columns}


\end{frame}
%---------------------------------------------------------------------------------------
%---------------------------------------------------------------------------------------
\begin{frame}
\frametitle{Expresiones para I(V) seg\'un BCS en los 3 casos}

\begin{columns}
\begin{column}{0.3\textwidth}
	\begin{figure}[!h] \label{fermi_levels}
	\includegraphics[width=\textwidth]{fermi_levels}
	\end{figure}
\end{column}
\begin{column}{0.7\textwidth}
\begin{flushleft}
	\begin{equation*}\label{inn}
		I^{NN} = \frac{4\pi e}{\hbar} |T|^2 N_1(\mu)N_2(\mu) eV
	\end{equation*}
	\begin{equation*}\label{cnn}
		C^{NN} = \frac{4\pi e}{\hbar} |T|^2 N_1(\mu)N_2(\mu) e
	\end{equation*}
\vspace{0.4cm}
	\begin{equation*}\label{ins}
		I^{NS} = \frac{C^{NN}}{e} \int_{-\infty}^{\infty} dE\ \frac{|E|}{\sqrt{E^2-\Delta^2}} [f(E-eV)-f(E)]
	\end{equation*}
\vspace{0.4cm}	
	\begin{equation*}\label{iss}
		I^{SS} = \frac{C^{NN}}{e} \int_{-\infty}^{\infty} dE\ 
		\frac{E^2\ [f(E-eV)-f(E)]}{\sqrt{(E^2-\Delta_1 ^2)}\sqrt{(E^2-\Delta_2 ^2)}}
	\end{equation*}
\end{flushleft}
\end{column}
\end{columns}

\end{frame}
%---------------------------------------------------------------------------------------
%---------------------------------------------------------------------------------------
\begin{frame}
\frametitle{Curva te\'oricas I(V) 1/2}

\begin{figure}[!h] \label{iv_teorico}
\includegraphics[width=\textwidth]{iv_teorico}
\end{figure}
	
\end{frame}
%---------------------------------------------------------------------------------------
%---------------------------------------------------------------------------------------
\begin{frame}
\frametitle{Curva te\'oricas I(V) 2/2}

\begin{figure}[!h] \label{iv_teorico2}
\includegraphics[width=\textwidth]{iv_teorico2}
\end{figure}

\end{frame}
%---------------------------------------------------------------------------------------
%---------------------------------------------------------------------------------------
\begin{frame}
\frametitle{Curva te\'oricas G(V)}

\begin{figure}[!h] \label{gv_teorico}
\includegraphics[width=\textwidth]{gv_teorico}
\end{figure}

\end{frame}
%---------------------------------------------------------------------------------------




%-----------------------------------------------------------------
%-----------------------------------------------------------------
\section{Experimental Method}
In order to measure the effect described above in the case of normal-superconductor tunneling, we prepare samples of two metals with different transition temperatures separated by an insulator. We vapor-deposited thin layers of aluminum and lead on a microscope glass slide,  leaving the Aluminum layer at the open air for a short period to let some insulating $AlO_2$ oxide form.

IT HAS TO BE CLEAR THAT WE HAVE A SANDWITCH!!!!!!!

1) sample preparation: vacuum chamber (torr?? why?? mean free path), filament, layer thickness (method for calculating it: isotropy or resistance? Justify that the second one gives smaller thickness with Poisson distribution, because we have rare events), ... HOW MUCH/MANY?????!!!!!!!!!!!!!!

2) Cryostat, nitrogen, helium, vacuum pump, manometer, T-P of vapor-pressure He, why don't we have different P's up and down in the cryostat? The T is different... And... HOW MUCH???

3) Measurement: 4 terminals (why? HOW MUCH?), constant steps sized intensity, 


%-----------------------------------------------------------------
%-----------------------------------------------------------------
\section{Results and Analysis}



\begin{figure}[h!]
\centering
\includegraphics[scale=0.4]{graph1}
\caption{Hola\label{graph1}}
\end{figure}

\begin{figure}[h!]
\centering
\includegraphics[scale=0.4]{graph2}
\caption{Hola\label{graph2}}
\end{figure}

\begin{figure}[h!]
\centering
\includegraphics[scale=0.4]{graph3}
\caption{Hola\label{graph3}}
\end{figure}

\begin{figure}[h!]
\centering
\includegraphics[scale=0.4]{graph4}
\caption{Hola\label{graph4}}
\end{figure}

\begin{figure}[h!]
\centering
\includegraphics[scale=0.4]{graph5}
\caption{Hola\label{graph5}}
\end{figure}



1) Levenberg-Marquard??? The best method: by hand... :-)
2) Graphs: commentary on ALL the characteristics...
3) BCS is not totally correct --> real density of states is not the BCS's one, phonons,
4) ...


These are the Experimental ResultsThese are the Experimental Results These are the Experimental Results These are the Experimental Results These are the Experimental Results These are the Experimental Results These are the Experimental Results These are the Experimental Results These are the Experimental Results These are the Experimental Results These are the Experimental Results These are the Experimental Results These are the Experimental Results These are the Experimental Results These are the Experimental Results These are the Experimental Results These are the Experimental Results These are the Experimental Results These are the Experimental Results These are the Experimental Results These are the Experimental Results 


%-----------------------------------------------------------------
%-----------------------------------------------------------------
\section*{Summary}
This method is very suitable to prove the general validity of the BCS theory, with statistical errors low enough to identify small discrepancies between real and theoretical densities of states. Numerical fitting with the Levenberg-Marquard method is not adequate because BCS theory is not entirely compatible with experimental data, and probably. However, fitting the data "by hand" turns out to be sufficiently accurate.


%-----------------------------------------------------------------
%-----------------------------------------------------------------
%-----------------------------------------------------------------
%-----------------------------------------------------------------
\begin{thebibliography}{99}

\bibitem{giaever1} I. Giaever and K. Megerle, \emph{Study of Superconductors by Electron Tunneling}, Phys. Rev. vol. 122, Num. 4, p. 1101 (1961)

\bibitem{giaever2} I. Giaever, \emph{Electron Tunneling and Superconductivity}, Rev. Mod. Phys. vol. 48, Num. 2, (1974)

\bibitem{giaever3} I. Giaever, \emph{Energy Gap in Superconductor Measured by Electron Tunneling}, Phys. Rev. Let. vol. 5, Num. 4, p. 147 (1960)

\bibitem{frerichs} R. Frerichs and J. P. Wilson, \emph{Tunnel-Effect Measurements on Superimposed Layers of Lead and Aluminum}, Phys. Rev. vol. 142, Num. 1, p. 264 (1966)

\bibitem{bcs} J. Bardeen, L. N. Cooper and J. R. Schrieffer, \emph{Theory of Superconductivity}, Phys. Rev. vol. 108, Num. 5, p. 1175 (1957)

\bibitem{films} M. Strongin, O. F. Kammerer \& A. Paskin, \emph{Superconducting Transition Temperature of Thin Films},  Phys. Rev. Lett. 14, p. 949 - 951 (1965)

\bibitem{recipes} W. H. Press, S. A. Teukolsky, W. T. Vetterling and B. P. Flannery, \emph{Numerical Recipes},  Cambridge University Press (1997)

\bibitem{roberts} T. R. Roberts \& S. G. Sydoriak, \emph{Thermodynamic Properties of Liquid Helium Three. Vapor Pressures below 1 K}, Phys. Rev. p. 106, 175 - 182 (1957) 

\bibitem{weber} S. Weber. \& G. Schmidt,  \emph{Experimentelle Untersuchungen Ober die Thermomolekulare Druckdifferenz in der Nahe der Grenzbedingung $P_1/P_2 = (T_1/T_2)^{1/2}$ und Vergleichung mit der Theorie}, Leiden Communication, 246C, p. 1-13 (1936) 

\end{thebibliography}

%-----------------------------------------------------------------
%-----------------------------------------------------------------
%-----------------------------------------------------------------
\ednotemessage
%-----------------------------------------------------------------
\end{document}
