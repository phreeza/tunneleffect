
\documentclass[twocolumn, twoside,a4paper,10pt]{article}
\pagestyle{plain}
\addtolength{\textheight}{2cm}

\usepackage{amsmath, amsthm, amsfonts}
\usepackage{graphicx}
%---------------------------------------------------------------------
\title{\textbf{Superconductivity and Electron Tunneling}}
\author{Thomas McColgan and Miguel Garc\'ia Echevarr\'ia\\
	\small{\textit{Low Temperature Laboratoy, Condensed Matter Physics Department}} \\
	\small{\textit{Faculty of Science, UAM}}
	}
\date{Madrid, \today}
%---------------------------------------------------------------------
\begin{document}
%-----------------------------------------------------------------
%-----------------------------------------------------------------
%ABSTRACT
\twocolumn[ 
\begin{@twocolumnfalse} 
\maketitle % need full-width title 
\begin{abstract} 
\small{Tunnel effect between metal layers is analyzed here. A potential difference imposed between two metal layers creates an electron tunneling current, and its relation with the potential depends on the state of the layers. Here is presented the case when one of the metals, Lead, is in the superconducting state and the other, Aluminum, in the normal state. By analysis of the data the gap of the Lead and other consequences can be inferred.}
\vspace{1cm}
\end{abstract}
\end{@twocolumnfalse}
]

%-----------------------------------------------------------------
%-----------------------------------------------------------------
\section{Introduction}
This is the Introduction This is the Introduction This is the Introduction This is the Introduction This is the Introduction This is the Introduction This is the Introduction This is the Introduction This is the Introduction This is the Introduction This is the Introduction This is the Introduction This is the Introduction This is the Introduction This is the Introduction This is the Introduction This is the Introduction This is the Introduction 

This is the Introduction This is the Introduction This is the Introduction This is the Introduction This is the Introduction This is the Introduction This is the Introduction This is the Introduction This is the Introduction This is the Introduction This is the Introduction This is the Introduction This is the Introduction This is the Introduction This is the Introduction This is the Introduction This is the Introduction This is the Introduction This is the Introduction This is the Introduction This is the Introduction This is the Introduction This is the Introduction This is the Introduction This is the Introduction This is the Introduction

%-----------------------------------------------------------------
%-----------------------------------------------------------------
\section{Theoretical Approach}
A LITTLE BIT OF MATHEMATICS FOR ELECTRON TUNNELING... FROM GIAEVER'S PAPER AND ANDRES' SOLID IV COURSE NOTES

The transition rate $W$ for an electron from an occupied state in the layer 1 with momentum $\mathbf{p_1}$ to a free state in the layer 2 with momentum $\mathbf{p_2}$ can be calculated by means of the Fermi Golden Rule like follows:
\begin{equation}\label{probability1}
W_{1\to 2}^{1e} = \frac{2\pi}{\hbar} |T_{21}|^2 f(\epsilon_{\mathbf{p_1}}) [1-f(\epsilon_{\mathbf{p_2}})]
		\delta(\epsilon_{\mathbf{p_1}}-\epsilon_{\mathbf{p_2}})\delta_{s_1s_2},
\end{equation}
where $T_{21}$ is the transition amplitud, $ f(\epsilon_{\mathbf{p_1}})$ the probability that the state $\mathbf{p_1}$ is occupied and  $[1-f(\epsilon_{\mathbf{p_2}})]$ the probability that the state $\mathbf{p_f}$ is empty, and the conservation of the energy and spin in the transition are assumed. 

The total transition rate from the electrode 1 to the electrode 2 will be
\begin{equation}\label{probability2}
W_{1\to 2} = \frac{2\pi}{\hbar} \sum_{\substack{s_1,s_2\\ \mathbf{p_1},\mathbf{p_2}}} |T_{21}|^2 
		f(\epsilon_{\mathbf{p_1}}) [1-f(\epsilon_{\mathbf{p_2}})] 
		\delta(\epsilon_{\mathbf{p_1}}-\epsilon_{\mathbf{p_2}})\delta_{s_1s_2}.
\end{equation}

If the transition hamiltonian does not depend on the spin and couples weakly the electrodes when the applied voltage is small (a valid approximation for the voltages used in this experiment), we can simplify the expression \eqref{probability2}:
\begin{equation}\label{probability3}
W_{1\to 2} = \frac{4\pi}{\hbar} |T|^2 \sum_{\mathbf{p_1},\mathbf{p_2}}  
		f(\epsilon_{\mathbf{p_1}}) [1-f(\epsilon_{\mathbf{p_2}})] 
		\delta(\epsilon_{\mathbf{p_1}}-\epsilon_{\mathbf{p_2}}).
\end{equation}
The transition rate in the opposite way is analogous.

Now we can write explicitly the expression for the current in the $1\to 2$ direction:
\begin{equation}\label{current1}
I = e\ (W_{1\to 2} - W_{2\to 1}),
\end{equation}
that is
\begin{equation}\label{current2}
I = \frac{4\pi e}{\hbar} |T|^2 \sum_{\mathbf{p_1},\mathbf{p_2}}  
		[f(\epsilon_{\mathbf{p_1}})-f(\epsilon_{\mathbf{p_2}})] 
		\delta(\epsilon_{\mathbf{p_1}}-\epsilon_{\mathbf{p_2}})..
\end{equation}

If we replace the summatories by integrals, considering that the momentums configure a quasicontinuum, and assuming a voltage difference $V$ between the electrodes that makes $\mu_2-\mu_1=eV$, we get
\begin{eqnarray}\label{current3}
&I& = \frac{4\pi e}{\hbar}  |T|^2 \times
	\nonumber \\
	&\times& \int_{-\infty}^{\infty} d\epsilon\ N_1(\epsilon-eV)\ N_2(\epsilon) [f(\epsilon-eV)-f(\epsilon)],
	\nonumber \\
\end{eqnarray}
where $N(E)$ is the density of states, needed to perform the change from summatories to the integral.

Now three cases can be distinguished, that is, when both electrodes are metals in the normal state, when only one of them is in the superconducting state and when both are in the superconducting state. The only difference between these situations is the form of the density of states $N(E)$, so it must be replaced by the adequate expression.

%-----------------------------------------------------------------
\subsection{Normal-Normal junction}
For a sufficient small voltage $V$, the state densities can be considered nearly constant, equation \eqref{current3} reads
\begin{equation}\label{inn}
I^{NN} = \frac{4\pi e}{\hbar} |T|^2 N_1(\mu)N_2(\mu) eV,
\end{equation}
where we have used the fact that $$ \int_{-\infty}^{\infty}d\epsilon [f(\epsilon-eV)-f(\epsilon)] \simeq eV. $$

From this equation can be derived easily the normal conductance of the junction
\begin{equation}\label{cnn}
C^{NN} = \frac{1}{R^{NN}} = \frac{dI^{NN}}{dV} = \frac{4\pi e^2}{\hbar} |T|^2 N_1(\mu)N_2(\mu).
\end{equation}


%-----------------------------------------------------------------
\subsection{Normal-Superconductor junction} 
The state density for a superconductor can be derived from considering a continuum spectrum of energy levels, and hence
\begin{equation}
N_N(\epsilon) d\epsilon = N_S(E)dE.
\end{equation}

The relation between $\epsilon$ and $E$ in the range of the BCS Theory of Superconductivity is $E_{\mathbf{p}} = \sqrt{\epsilon_{\mathbf{p}}^2 + \Delta^2}$, with $\Delta$ the gap of the superconductor. So we can get
\begin{equation}\label{ns}
N_S(E) = N_N(\epsilon) \left | \frac{d\epsilon}{dE} \right | =  
\left\{ 
\begin{array}{ll} 
N_N(\epsilon)\frac{|E|}{\sqrt{E^2-\Delta^2}},	&	|E| > \Delta 	\\ 
0,								& 	|E| \geq \Delta	\\
\end{array}
\right..
\end{equation}

If we replace this state density in \eqref{current3} we get the $I^{NS}$, but only up to $|T|^2$ order. There are high order effects by means of Cooper pair transmission to the superconducting electrode. Neglecting these issues, the current for small voltages is
\begin{eqnarray}\label{ins_previous}
&I^{NS}& = \frac{4\pi e}{\hbar} |T|^2 N_{1N}(\mu) \times
		\nonumber \\
		&\times& \int_{-\infty}^{\infty} dE\ N_{2S}(E) [f(E-eV)-f(E)] =
		\nonumber \\
		&=& \frac{4\pi e}{\hbar} |T|^2 N_{1N}(\mu) N_{2N}(\mu) \times
		\nonumber \\
		&\times& \int_{-\infty}^{\infty} dE\ \frac{|E|}{\sqrt{E^2-\Delta^2}} [f(E-eV)-f(E)].
		\nonumber \\
\end{eqnarray}

It can be expressed in terms of the $C^{NN}$ like follows:
\begin{equation}\label{ins}
I^{NS} = \frac{C^{NN}}{e} \int_{-\infty}^{\infty} dE\ \frac{|E|}{\sqrt{E^2-\Delta^2}} [f(E-eV)-f(E)].
\end{equation}

Finally, introducing $x=E-\epsilon$ and noting that Fermi functions are even, we get the expression that is used for numerical analysis:
\begin{eqnarray}\label{ins_numerical}
I^{NS} &=& \frac{C^{NN}}{e} \int_{0}^{\infty} dx\ \frac{x+\epsilon}{\sqrt{x(x+2\epsilon)}} \times
		\nonumber \\
		&\times& [f(x+\epsilon-eV)-f(x+\epsilon+eV)].
		\nonumber \\
\end{eqnarray}

From eqref{ins}, the conductance will be the following
\begin{eqnarray}\label{cns}
C^{NS} &=& \frac{1}{R^{NS}} = \frac{dI^{NS}}{dV} =
		\nonumber \\
		&=& \frac{C^{NN}}{e} \int_{-\infty}^{\infty} dE\ \frac{|E|}{\sqrt{E^2-\Delta^2}} 
		\frac{\partial f(E-eV)}{\partial V}
		\nonumber \\
\end{eqnarray}

%-----------------------------------------------------------------
\subsection{Superconductor-Superconductor junction} 
By analogy with the previous section, we write directly the expression of the current for this situation:
\begin{equation}\label{ins}
I^{SS} = \frac{C^{NN}}{e} \int_{-\infty}^{\infty} dE\ 
		\frac{E^2\ [f(E-eV)-f(E)]}{\sqrt{E^2-\Delta_1^2}\sqrt{E^2-\Delta_2^2}}.
\end{equation}


%-----------------------------------------------------------------
%-----------------------------------------------------------------
\section{Experimental Method}
This is the Experimental Method This is the Experimental Method This is the Experimental Method This is the Experimental Method This is the Experimental Method This is the Experimental Method This is the Experimental Method This is the Experimental Method This is the Experimental Method This is the Experimental Method This is the Experimental Method This is the Experimental Method This is the Experimental Method This is the Experimental Method This is the Experimental Method This is the Experimental Method This is the Experimental Method This is the Experimental Method This is the Experimental Method This is the Experimental Method This is the Experimental Method This is the Experimental Method This is the Experimental Method This is the Experimental Method This is the Experimental Method This is the Experimental Method This is the Experimental Method This is the Experimental Method This is the Experimental Method This is the Experimental Method This is the Experimental Method This is the Experimental Method This is the Experimental Method This is the Experimental Method This is the Experimental Method This is the Experimental Method This is the Experimental Method This is the Experimental Method This is the Experimental Method This is the 

Experimental Method This is the Experimental Method This is the Experimental Method This is the Experimental Method This is the Experimental Method This is the Experimental Method This is the Experimental Method This is the Experimental Method This is the Experimental Method This is the Experimental Method This is the Experimental Method This is the Experimental Method This is the Experimental Method This is the Experimental Method This is the Experimental Method This is the Experimental Method This is the Experimental Method This is the Experimental Method This is the Experimental Method This is the Experimental Method This is the Experimental Method This is the Experimental Method This is the Experimental Method This is the Experimental Method This is the Experimental Method This is the Experimental Method This is the Experimental Method This is the Experimental Method This is the Experimental Method This is the Experimental Method This is the Experimental Method This is the Experimental Method This is the Experimental Method This is the Experimental Method This is the Experimental Method This is the Experimental Method This is the Experimental Method This is the Experimental Method This is the Experimental Method This is the Experimental Method This is the Experimental Method This is the Experimental Method This is the 

%-----------------------------------------------------------------
%-----------------------------------------------------------------
\section{Experimental Results}
These are the Experimental Results These are the Experimental Results These are the Experimental Results These are the Experimental Results These are the Experimental Results These are the Experimental Results These are the Experimental Results These are the Experimental Results These are the Experimental Results These are the Experimental Results These are the Experimental Results These are the Experimental Results These are the Experimental Results These are the Experimental Results These are the Experimental Results These are the Experimental Results These are the Experimental Results These are the Experimental Results These are the Experimental Results These are the Experimental Results These are the Experimental Results These are the Experimental Results These are the Experimental Results These are the Experimental Results  
 
These are the Experimental ResultsThese are the Experimental Results These are the Experimental Results These are the Experimental Results These are the Experimental Results These are the Experimental Results These are the Experimental Results These are the Experimental Results These are the Experimental Results These are the Experimental Results These are the Experimental Results These are the Experimental Results These are the Experimental Results These are the Experimental Results These are the Experimental Results These are the Experimental Results These are the Experimental Results These are the Experimental Results These are the Experimental Results These are the Experimental Results These are the Experimental Results 


%-----------------------------------------------------------------
%-----------------------------------------------------------------
%-----------------------------------------------------------------
%-----------------------------------------------------------------
\begin{thebibliography}{99}

\bibitem{giaever1} I. Giaever and K. Megerle, \emph{Study of Superconductors by Electron Tunneling}, Phys. Rev. vol. 122, Num. 4, p. 1101 (1961)

\bibitem{giaever2} I. Giaever, \emph{Electron Tunneling and Superconductivity}, Rev. Mod. Phys. vol. 48, Num. 2, (1974)

\bibitem{giaever3} I. Giaever, \emph{Energy Gap in Superconductor Measured by Electron Tunneling}, Phys. Rev. Let. vol. 5, Num. 4, p. 147 (1960)

\bibitem{frerichs} R. Frerichs and J. P. Wilson, \emph{Tunnel-Effect Measurements on Superimposed Layers of Lead and Aluminum}, Phys. Rev. vol. 142, Num. 1, p. 264 (1966)

\bibitem{bcs} J. Bardeen, L. N. Cooper and J. R. Schrieffer, \emph{Theory of Superconductivity}, Phys. Rev. vol. 108, Num. 5, p. 1175 (1957)

\bibitem{ashcroft} N. W. Ashcroft and N. D. Mermin, \emph{Solid State Physics}, Harcourt College Publishers (1976)

\bibitem{formulas} Jianping Li, \emph{General explicit difference formulas for numerical differentiation}, Journal of Computational and Applied Mathematics vol. 183, p. 29 (2005)

\bibitem{recipes} W. H. Press, S. A. Teukolsky, W. T. Vetterling and B. P. Flannery \emph{Numerical Recipes},  Cambridge University Press (1997)



\end{thebibliography}

%-----------------------------------------------------------------
%-----------------------------------------------------------------
%-----------------------------------------------------------------
%-----------------------------------------------------------------
\end{document}