%\frame
%{
%  \frametitle{Features of the Beamer Class}

%  \begin{itemize}
%  \item<1-> Normal LaTeX class.
%  \item<2-> Easy overlays.
%  \item<3-> No external programs needed.      
%  \end{itemize}
%}


%Numerical Integration/Differentiation

\frame
{
  \frametitle{Integracion y Differenciacion Numerica}
   
   \begin{itemize}
     \item<1-> Derivamos ajustando una recta a $n$ puntos consecutivos. La pendiente aproxima la derivada.
     \item<2-> Este methodo tambien suaviza los datos.
     \item<3-> Nunca tenemos $\frac{\Delta I}{\Delta V}=0$ porque $\Delta I>0$
     \item<4-> Tambien tenemos que integrar
     \begin{equation*}\label{ins}
		I^{NS} = \frac{C^{NN}}{e} \int_{-\infty}^{\infty} dE\ \frac{|E|}{\sqrt{E^2-\Delta^2}} [f(E-eV)-f(E)]
	\end{equation*}
     \item<5-> Lo hacemos analiticamente para $|E|$ peque\~nos y numericamente para $|E|\to\infty$
   \end{itemize}
  
}

%Discrepancies


\frame
{
  \frametitle{Differencias al Modelo}
  
  \begin{itemize}
  \item<1-> El modelo BCS no predica perfectamente la densidad de estados
  \item<2-> La causa puede ser que hay interacion phonon-electron adicional
  \item<3-> Giaever tambien observi\'o este efecto.
  \item<4-> Debido a esta diferencia no podemos ajustar los datos automaticamente.
  \end{itemize}
  
  
}

%Calculating T and C


\frame
{
  \frametitle{Calculando Temperatura y Conductancia}
  
    \begin{itemize}
     \item<1-> Tenemos 3 parametros independientes: $T$,$C_{NN}$ y $\Delta$ 
     \item<2-> Podemos hallar $T$ y $C_{NN}$ con otras medidas
     \item<3-> Calculamos $C_{NN}$ usando medidas de conductancia a $T\gg T_C$
     \item<4-> Calculamos $T$ usando las medidas de presion, y la presion vapor de $He$.
  \end{itemize}}

%Fitting gap by hand


\frame
{
  \frametitle{Ajustando el Gap}
  
      \begin{itemize}
     \item<1-> Ahora podemos ajustar $\Delta$, el gap.
     \item<2->  $\Delta$ cambia poco($0.1 meV$) en nuestro rango de temperaturas .
     \item<3-> Intentamos ajustar un valor de  $\Delta$ para todas medidas.
     \item<4-> Encontramos un valor de $\Delta = (1.4 \pm 0.1) meV$
     \end{itemize}
}

%Results

%Low Temp measurement

\frame
{
  \frametitle{Temperaturas mas bajas}
     \begin{itemize}
      \item<1-> Nuestra ultima medida tenia $T\approx 1.2K$
      \item<2-> Hemos encontrado una curva de conductancia muy distincta.
      \item<3-> Se puede explicar con la transicion de partes del $Al$ a superconductancia.
      \item<4-> No hay conductancia negativa, asi que no esta totalmente superconductor.
     \end{itemize}
  
}