%---------------------------------------------------------------------------
%---------------------------------------------------------------------------
\frame
{
  \frametitle{Preparaci\'on de la Muestra}
\begin{columns}
\begin{column}{0.6\textwidth}
   \begin{itemize}
      \item<1-> Sandwich de $Al$,$Al_2O_3$ y $Pb$
      \item<2-> $T_c(Pb)=7.193K$, $T_c(Al)=1.140K$ $\to$ uni\'on Superconductor-Normal
      \item<3-> $Al_2O_3$ es el aislante $\to$ barrera t\'unel.
      \item<4-> Preparaci\'on por evaporaci\'on en alto vac\'io ($10^{-6}<p(torr)<10^{-3}$).
      \item<5-> Tenemos varias  uniones en cada cristal ($\sim 5$)
  \end{itemize}
  \end{column}
\begin{column}{0.4\textwidth}
	\begin{figure}[!h] \label{sample}
	\includegraphics<4->[width=\textwidth]{sample}
	\end{figure}
\end{column}

\end{columns}
}

%---------------------------------------------------------------------------
%---------------------------------------------------------------------------
\frame
{
  \frametitle{Estimando las Anchuras de Capas}
  
  \textcolor{blue}{Por qu\'e?} \hspace{1cm} $\longrightarrow$ \hspace{1cm}
  Anchura $\leftrightarrow$ Resistencia (Conductancia) 
  
  \pause
  \begin{itemize}
     \item<1-> \textcolor{blue}{C\'omo?} $\to$ 2 M\'etodos para hacerlo:
        
        \begin{enumerate}
           \item<2-> Ley de Ohm: $h=\frac{\rho l}{Rb}$
           \item<3-> Suponiendo evaporaci\'on isotropa: $h=\frac{V}{4\pi r}$
        \end{enumerate}
     \item<4-> \textcolor{red}{Difieren en un orden de magnitud!! Cu\'al est\'a bien??}   
     \item<5-> Es correcto el segundo. El primero no funciona porque la anchura no es suficientemente homog\'enea.
    \end{itemize}
     
    
\uncover<6->{Adem\'as la anchura (resistencia) del aislante se controla por tiempo de exposici\'on al aire (asint\'otico)
     \begin{figure}[!h] \label{sample}
	\includegraphics<6->[width=0.5\textwidth]{isolator}
     \end{figure}
     }
}

%---------------------------------------------------------------------------
%---------------------------------------------------------------------------
\frame
{
  \frametitle{Enfriando la Muestra}
   \begin{columns}
   \begin{column}{0.6\textwidth}

   \begin{itemize}
     \item<1->Necesitamos $T$ entre $T_{C,Pb}=7.193 K$ y $T_{C,Al}=1.140 K$
     \item<2->Ponemos la muestra en un criostato con $He$ l\'iquido,
     \item<3->A $p_{\text{ambiente}}$ $He$ l\'iquido tiene $T=4.2 K$
     \item<4->Podemos bajar $T$ bajando la presi\'on en el criostato
   \end{itemize}
   
     \end{column}
     \begin{column}{0.4\textwidth}
	\begin{figure}[!h] \label{sample}
	\includegraphics<1-3>[width=\textwidth]{cryostat}
	\includegraphics<4->[width=\textwidth]{vap_he}
     \end{figure}
\end{column}
\end{columns}
}

%---------------------------------------------------------------------------
%---------------------------------------------------------------------------  
\frame
{
  \frametitle{Tomando los Datos}
     \begin{columns}
   \begin{column}{0.6\textwidth}

  \begin{itemize}
    \item<1-> S\'olo nos interesa la conductancia de la uni\'on, no la de los cables
    \item<2-> Por eso medimos con 4 terminales
    \item<3-> Medimos a intervalos fijos de intensidad
    \item<4-> Rangos y resoluciones diferentes.
  \end{itemize}
  
       \end{column}
     \begin{column}{0.4\textwidth}
	\begin{figure}[!h] \label{sample}
	\includegraphics<2->[width=\textwidth]{4term}
     \end{figure}
\end{column}
\end{columns}

}